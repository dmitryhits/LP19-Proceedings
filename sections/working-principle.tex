\subsection{Working Principle}
Its basic principle is shown in Figure \vref{fig:3d1}: In a planar detector 
\fig{.2}{3DConcept.pdf}[Comparison of the planar and the 3D detector concepts.][fig:3d1]
the readout and bias electrodes are brought onto the front and back side of the sensor with a thickness $D$. The resulting drift distance $L$ of the charge carriers is of the order of $D$. In the 3D detector the electrodes are put inside of the detector material so that the \acp{MIP} can travel the same distance $D$ in the material and therefore create the same amount charge carriers but $L$ is largely reduced.
