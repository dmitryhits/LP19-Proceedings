\subsection{Fabrication}
In order to manufacture the electrodes in diamond, columns were fabricated using a \SI{130}{\femto\second} laser with a wavelength of \SI{800}{\nano\meter} which converts the diamond into a electrically resistive mixture of different carbon phases \cite{3dfab}. A \ac{SLM} was used to correct aberrations during fabrication to achieve a column yield of $\gtrsim$\SI{ 99.8}{\%}, a column diameter of \SI{2.6}{\micro\meter} and a resistivity of the columns of the order of \SIrange{.1}{1}{\ohm\cm} \cite{slm}. The columns are not drilled completely through the diamond, but with a gap of \SI{15}{\micro\meter} to the opposite surface to avoid high voltage breakthrough. So far the largest fabricated device has about 4000 3D cells, where one cell consists of four bias electrodes and one readout electrode in the centre. Thus \SI{\sim 7000}{columns} had to be drilled to build such a device. \par
The detector was constructed by connecting the bias columns with a metallisation on the bottom surface and the readout columns with a metallisation on the top surface. The sensor was then bump bonded to the readout electronics as shown in Figure \vref{fig:bb}. The detectors described herein were connected to two different \acp{ROC}, which is why different bonding processes, as shown in Figures \vref{fig:bbc} and \vref{fig:bba}, had to be developed. For both of these detectors a 3D cell size of \SI{50x50}{\micro\meter} was chosen. Since the layout of the \acp{ROC} has a different pixel pitch several cells were ganged together. Photographs of the assembled 3D detectors on the \acp{ROC} are shown in Figure \vref{fig:3d}.
\subfigs{\subfig{.12}{BondingSchemeCMS.pdf}[CMS PSI46dig][fig:bbc]}{\subfig{.12}{BondingSchemeATLAS.pdf}[ATLAS FE-I4B][fig:bba]}[Bump Bonding and metallisation for two different \acp{ROC}.][fig:bb]
\subfigs{\subfig{.18}{3DCMS.jpg}[CMS PSI46dig][fig:3dc]}{\subfig{.18}{3DATLAS.jpg}[ATLAS FE-I4B][fig:3da]}[Fully assembled 3D pixel detectors.][fig:3d]