\subsection{Fabrication}
All devices discussed in this article were constructed with \ac{pCVD} diamond. The sensors are thin plates with a thickness of \SI{\sim 500}{\um} and a side length of \SI{\sim 5}{\mm}. In order to manufacture the electrodes in diamond, columns were fabricated perpendicular to the large side using a \SI{130}{\femto\second} laser with a wavelength of \SI{800}{\nano\meter}. After focusing to a \SI{2}{\um} spot the laser has the energy density to convert diamond into a electrically resistive mixture of different carbon phases \cite{3dfab}. A \ac{SLM} \cite{slm} was used to correct spherical aberrations during fabrication. This helped to achieve the high column yield of $\gtrsim$\SI{ 99.8}{\%}, a column diameter of \SI{2.6}{\micro\meter} and a resistivity of the columns of the order of \SIrange{.1}{1}{\ohm\cm}. The yield was measured by manually counting missing or broken columns and the diameter was optically measured. The columns are not drilled completely through the diamond, but with a gap of \SI{15}{\micro\meter} to the opposite surface to avoid high voltage breakthrough at the operated bias voltages of up to \SI{70}{\volt}. The detector was constructed by ganging all bias columns together with a mesh metallisation on one surface and metallising the other surface to connect a small number of readout columns in order to match the pitch of the readout electronics for bump bonding.\par
The detectors described herein were connected to two different \acp{ROC}, which is why different bonding processes, as shown in Figures \ref{fig:bb}, were used. For both of these detectors a 3D cell size of \SI{50x50}{\micro\meter} was chosen. Since the layout of the \acp{ROC} had a different pixel pitch several cells were ganged together by connecting several readout columns with the surface metallisation. Detector 1 was connected to the PSI46digV2.1respin \ac{ROC} \cite{kornmayer} with a \SI{3x2}{} cell ganging to match the pixel pitch of \SI{150x100}{\micro\meter}. It was bump bonded to the \ac{ROC} at the Nanofabrication Lab at the Princeton University with indium. This was achieved by putting equal height indium columns on both \ac{ROC} and the sensor and then pressing them together without reflow (see Figure \ref{fig:bbc}). The second detector was connected to the FE-I4B \ac{ROC} \cite{fei4} with a \SI{5x1}{} cell ganging due to the pixel pitch of \SI{250x50}{\micro\meter}. The bump bonding for this sensor was performed at IFAE-CNM in Barcelona by an adapted process with tin-silver bumps (see Figure \ref{fig:bba}).\par
\subfigs{\subfig{.12}{BondingSchemeCMS.pdf}[CMS PSI46dig][fig:bbc]}{\subfig{.12}{BondingSchemeATLAS.pdf}[ATLAS FE-I4B][fig:bba]}[Bump Bonding and metallisation for two different \acp{ROC}.][fig:bb]
Both devices have \SI{\sim 3500}{} 3D cells, where one cell consists of four bias electrodes and one readout electrode in the centre. Since the bias electrodes are shared between the cells, \SI{\sim 7200}{columns} had to be drilled to build such a device. Except for the different readout, the sensors are very similar. Photographs of the assembled 3D detectors on the \acp{ROC} are shown in Figure \ref{fig:3d}. 
% \small{
%   \nicetab{|l|c|c|}{    & \textbf{1}                    & \textbf{2}                    \\\hline
%   readout chip (ROC)    & PSI46digv2.1respin            & FE-I4B                        \\
%   pixel pitch           & \SI{150x100}{\micro\meter}    & \SI{250x50}{\micro\meter}     \\
%   3D cell size          & \SI{50x50}{\micro\meter}      & \SI{50x50}{\micro\meter}      \\
%   ganging               & \SI{3x2}{}                    & \SI{5x1}{}                    \\
%   size                  & \SI{4.85x4.90}{\milli\meter}  & \SI{4.90x4.94}{\milli\meter}  \\
%   thickness             & \SI{500}{\micro\meter}        & \SI{510}{\micro\meter}        \\
%   \SI{50x50}{pixels}    & \SI{67x53}{}                  & \SI{53x67}{}                  \\
%   3D columns            & 7223                          & 7223                          \\
%   column diameter       & \SI{2.6}{\micro\meter}        & \SI{2.6}{\micro\meter}        \\
%   active area           & \SI{3.45x3.19}{\milli\meter}  & \SI{3.2x3.5}{\milli\meter}    \\
%   bump bonding          & tin silver (IFAE)             & indium (Princeton)            \\
%   }[Properties of the 3D diamond detectors.][tab:dev]}
\subfigs{\subfig{.18}{3DCMS.jpg}[Detector 1 with \SI{3x2}{} ganging.][fig:3dc]}{\subfig{.18}{3DATLAS.jpg}[Detector 2 with \SI{5x1}{} ganging.][fig:3da]}[Assembled 3D pixel detectors.][fig:3d]