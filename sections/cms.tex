\subsection{PSI46digV2.1respin readout}
The first 3D pixel detector prototype was connected to the PSI46digV2.1respin \ac{ROC} \cite{kornmayer} with a \SI{3x2}{} cell ganging to match the pixel pitch of \SI{150x100}{\micro\meter}. The 3D sensors were bump bonded to the \ac{ROC} at the Nanofabrication Lab at the Princeton University with indium. This was achieved by putting equal height indium columns on both \ac{ROC} and the sensor and then pressing them together.\par
This detector was tested both in beam lines at \ac{PSI} and the SPS facility at CERN. The preliminary beam test results show that, relative to a planar silicon device, the efficiency within a selected fiducial area was \SI{99.2}{\%}. Where the hit efficiency was defined as the percentage of hits in the 3D pixel detector when a particle track traversed the detector. This exact value was measured at both beam test facilities. The fiducial area was selected such as to exclude non-working 3D cells found by visual inspection, which can happen due to broken or missing columns or due to metallisation issues. A small mismatch between a 3D and a planar device is expected due to a region inside the cells where the electric field is low \cite{guilio} and due to the relative inefficiency of the columns themselves. The area of the 3D columns compared to the whole cell is \SI{.4}{\%}. \par 
Figure \vref{fig:evc} shows that the efficiency of the detector plateaus at a voltage of \SI{30}{\volt}. The measurements at both facilities also agree well. This demonstrates that the device already operate well at comparably low voltages.\par 
\subfigs{\subfig{.2}[r]{EffVolB6P.pdf}[PSI.][fig:evcc]}{\subfig{.2}[r]{EffVolB6C.pdf}[CERN SPS.][fig:evcp]}[Hit efficiency vs. bias voltage.][fig:evc]
The preliminary analysis of the pulse height distribution for the measurements at CERN yields a mean value of \SI{\sim 14}{\kilo e}. The precise pulse height calibration of the \ac{ROC} is currently being studied.