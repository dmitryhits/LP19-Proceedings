\subsection{Results of Detector 1 (3$\times$2 cell ganging)}
This detector was tested with pixel telescopes in beam lines at \ac{PSI} and the SPS facility at CERN in order to get both measurements at high rates as well as with high tracking resolution. At \ac{PSI} a beam of \SI{210}{\mev/c} positive pions was used \cite{pim1}, where protons were removed with a plastic absorber. Whereas at CERN the momentum of the positive hadron beam was \SI{120}{\gev/c} \cite{h6}. The preliminary beam test results show that, relative to a planar silicon device, the efficiency within a selected fiducial area was \SI{99.2}{\%}. Where the hit efficiency was defined as the percentage of hits in the 3D pixel detector when a particle track traversed the detector. This value was measured at both beam test facilities. The fiducial area was selected such as to exclude non-working 3D cells found by visual inspection, which can happen due to broken or missing columns or due to metallisation issues, where the metal is not properly connected to the electrodes. A small discrepancy between a 3D and a fully efficient planar device is partly expected due to the relative inefficiency of the columns themselves. The area of the 3D columns compared to the whole cell is \SI{.4}{\%}. \par 
Figure \vref{fig:evc} shows that the efficiency of the detector plateaus at a voltage of \SI{30}{\volt}. This demonstrates that the device already operates well at very low voltages compared to a planar diamond detector. The efficiency at both facilities agrees well, which shows that the detector is independent of the different particle energies and intensities which were used.\par 
\subfigs{\subfig{.2}[r]{EffVolB6P.pdf}[PSI.][fig:evcc]}{\subfig{.2}[r]{EffVolB6C.pdf}[CERN SPS.][fig:evcp]}[Hit efficiency vs. bias voltage.][fig:evc]
The preliminary analysis of the pulse height distribution for the measurements at CERN yields a mean value of \SI{\sim 14}{\kilo e}. The precise pulse height calibration of the \ac{ROC} is currently being studied.