\abstract{
The planned upgrade of the LHC to the High-Luminosity-LHC will push the luminosity limits above the original design values. Since the current detectors will not be able to cope with this environment ATLAS and CMS are doing research to find more radiation tolerant technologies for their innermost tracking layers. Chemical Vapour Deposition (CVD) diamond is an excellent candidate for this purpose. Detectors out of this material are already established in the highest irradiation regimes for the beam condition monitors at LHC. The RD42 collaboration is leading an effort to use CVD diamonds also as sensor material for the future tracking detectors. The signal behaviour of highly irradiated diamonds is presented as well as the recent study of the signal dependence on incident particle flux. There is also a recent development towards 3D detectors and especially 3D detectors with a pixel readout based on diamond sensors.
}