\section{Introduction}
%%%%%%%%%%%%%%%%%%%%%%% MOTIVATION %%%%%%%%%%%%%%%%%%%%%%%%%%
The radiation levels of the \ac{HL-LHC} are expected to be a big challenge for the future detectors. By 2028 experiments must be prepared for an instantaneous luminosity of \SI{7.5e34}{\per\centi\meter\squared\per\second}. In this environment the innermost tracking layer at a transverse distance of \SI{\sim30}{\milli\meter} to the \acl{IP} will be exposed to a total fluence of \SI{2e16}{n_{eq}\per\centi\meter^2} %which corresponds to a total dose of the \orderof{\SI{10}{\mega\gray}} 
\cite{dose}. The expected lifetime of the current planar silicon tracking detectors would be about one year in such and environment.\par
\ac{CVD} diamond is investigated by the RD42 collaboration as a possible detector material \cite{rd42}. Its displacement energy of \SI{42}{\ev\per atom} makes it intrinsically radiation tolerant and the band gap of \SI{5.5}{\ev} greatly simplifies the construction of the detectors as well as guarantees negligible leakage currents. Compared to analogous silicon detectors, it was shown that diamond is at a minimum three times more radiation tolerant \cite{deboer}, collects the charges at least two times faster \cite{pernegger} and conducts heat four times more efficiently \cite{zhao}.\par
%%%%%%%%%%%%%%%%%%%%%%% INTRODUCTION %%%%%%%%%%%%%%%%%%%%%%%%%%
By now the technology of diamond detectors is well established in high energy physics. Many high energy physics experiments are already using \aclp{BCM} or \aclp{BLM} based on \ac{CVD} diamonds \cite{babar, bcm, dbm1}.\par
After the doses expected in the \ac{HL-LHC}, all detector materials become trap limited with a schubweg, the average drift distance before a free charge carrier gets trapped, below \SI{75}{\micro\meter}. 
Therefore RD42 collaboration is studying a novel detector design in diamond, namely 3D detectors. This detector design places column-like electrodes inside the detector material. Therefore the drift distance an electron-hole pair must undergo to reach an electrode can be reduced below the schubweg of an irradiated sensor without reducing the amount of created electron-hole pairs.
