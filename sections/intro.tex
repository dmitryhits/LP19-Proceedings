\section{Introduction}
%%%%%%%%%%%%%%%%%%%%%%% MOTIVATION %%%%%%%%%%%%%%%%%%%%%%%%%%
The radiation levels of the \ac{HL-LHC} are expected to be a big challenge for the future detectors. By 2028 experiments must be prepared for an instantaneous luminosity of \SI{7.5e34}{\per\centi\meter\squared\per\second}. In this environment the innermost tracking layer at a transverse distance of \SI{\sim30}{\milli\meter} to the \acl{IP} will be exposed to a total fluence of \SI{2e16}{n_{eq}\per\centi\meter^2} %which corresponds to a total dose of the \orderof{\SI{10}{\mega\gray}} 
\cite{dose}. The expected lifetime of the current planar silicon tracking detectors would be about one year in such an environment.\par
In this work, we present the test beam measurements of 3D detectors using \ac{pCVD} diamond, which were fabricated by the CERN RD42 collaboration \cite{rd42}. Diamond has a large minimum lattice displacement energy of \SI{43}{\ev\per atom} \cite{koike} which makes it intrinsically radiation tolerant and its band gap of \SI{5.5}{\ev} greatly simplifies the construction of the detectors as well as guarantees negligible leakage currents.\par
% Compared to analogous silicon detectors, it was shown that diamond is at a minimum three times more radiation tolerant \cite{deboer}, collects the charges at least two times faster \cite{pernegger} and conducts heat four times more efficiently \cite{zhao}.\par
%%%%%%%%%%%%%%%%%%%%%%% INTRODUCTION %%%%%%%%%%%%%%%%%%%%%%%%%%
After the doses expected in the \ac{HL-LHC}, all detector materials will be trap limited with a schubweg, the average drift distance before a free charge carrier gets trapped, below \SI{75}{\micro\meter} \cite{feick, wen2}. The RD42 collaboration is studying a novel detector design in diamond, namely 3D detectors, to extend the radiation tolerance of diamond to fluences exceeding the HL-LHC doses. This detector design places column-like electrodes inside the detector material. In this detector geometry the drift distance an electron-hole pair must undergo to reach an electrode can be reduced below the schubweg of an irradiated sensor without reducing the number of electron-hole pairs created. The details of the general working principles of 3D detectors may be found in \cite{parker}.
