\subsection{FE-I4B readout}
The second prototype was connected to the FE-I4B \ac{ROC} \cite{fei4} with a \SI{5x1}{} cell ganging due to the pixel pitch of \SI{250x50}{\micro\meter}. The bump bonding was performed at IFAE-CNM in Barcelona by an adapted process with tin-silver bumps. Using a high resolution beam telescope, with a spatial resolution of \SI{3}{\micro\meter} at the device under test, the efficiency could be mapped to the spatial coordinates. \par
The first results are shown in Figure \vref{fig:ea}. The analysis yields an efficiency of \SI{98.2}{\%} in the contiguous fiducial area (Figure \vref{fig:ema}). As shown in Figure \vref{fig:eva}, the detector reaches the same efficiency for all tested voltages. The efficiency being lower than \SI{99}{\%} is most likely due to issues with the bump bonding or the metallisation as indicated by the inefficient patches in the efficiency map. The preliminary pulse height in the fiducial region was \SI{\sim 14}{\kilo e} which is consistent with the result of the first prototype. The precise pulse height calibration for the FE-I4B \ac{ROC} is in the process of being performed.
\subfigs{\subfig[.45][.025]{.2}{EffMapB5.pdf}[Hit efficiency map. The red box denotes the fiducial area.][fig:ema]}{\subfig{.2}{EffVolB5.pdf}[Hit efficiency vs. voltage.][fig:eva]}[Results of the FE-I4 readout.][fig:ea]