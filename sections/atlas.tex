\subsection{FE-I4B readout}
This device was solely test at the CERN SPS beam line using a high resolution beam telescope, with a spatial resolution of \SI{3}{\micro\meter} at the device under test so that the efficiency could be mapped to the spatial coordinates. \par
The first results are shown in Figure \ref{fig:ea}. The analysis yields an efficiency of \SI{98.2}{\%} in the contiguous fiducial area (Figure \ref{fig:ema}). As shown in Figure \ref{fig:eva}, the detector reaches the same efficiency for all tested voltages. The efficiency being lower than \SI{99}{\%} is most likely due to issues with the bump bonding or the metallisation as suggested by the inefficient patches in the efficiency map. The preliminary pulse height in the fiducial region was \SI{\sim 14}{\kilo e} which is consistent with the result of the first prototype. The precise pulse height calibration for the FE-I4B \ac{ROC} is in the process of being performed.
\subfigs{\subfig[.45][.025]{.2}{EffMapB5.pdf}[Hit efficiency map. The red box denotes the fiducial area.][fig:ema]}{\subfig{.2}{EffVolB5.pdf}[Hit efficiency vs. voltage.][fig:eva]}[Results of the FE-I4 readout.][fig:ea]